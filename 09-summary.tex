\section{Summary}
For all the benefits it provides, DeFi faces a pressing challenge. Alongside legions of legitimate users, there are is no doubt bad actors such as criminal/terrorist groups and insider traders who are currently abusing DeFi protocols for malicious purposes. This is possible because the DeFi ecosystem currently lacks the KYC and AML checks built into centralized financial systems.

Further, the lack of KYC/AML hurts the DeFi community by preventing regulated market makers (and all the liquidity they could provide) from joining the ecosystem.

DeFi regulation is imminent and unavoidable—and there is a pressing need to implement a permissioning layer that supports KYC/AML while protecting the “spirit of DeFi.” This white paper has introduced the Gateway Protocol—an ecosystem that will provide this permissioning layer, allowing dApps and users to interact with privacy while delivering much-needed KYC and AML checks.

Top learning points from this white paper include:

\begin{itemize}
\item The Gateway Protocol is a cross-chain oracle token model that enables dApps to easily add a permissioning layer for a predetermined set of requirements.
\item User verification is completed by a Gatekeeper—a legal entity that provides the compliance, KYC, or identity verification service a dApp needs to comply with applicable regulations.
\item Gatekeepers belong to one or more Gatekeeper networks. Each network will enforce a specific set of requirements, which may be modeled on a set of trading regulations, e.g., those active in a DEX user’s country of residence.
\item Once a Gatekeeper has verified a User’s identity and conformance with the necessary requirements, it will issue them with a Gateway Pass.
\item dApps can operate with certainty simply by checking each User’s wallet for an active Gateway Pass issued by an appropriate Gatekeeper network.
\item When interacting with a participating dApp, users without a valid Pass will be directed to an appropriate Gatekeeper network for verification before any trades can be completed.
\item To retain the “spirit of DeFi,” the Gateway Protocol will operate as a DAO and be managed and maintained by all stakeholders—not a centralized institution.
\item Incentives and penalties exist to ensure stakeholders—particularly Gatekeepers—operate ethically and in the best interests of the Gateway Protocol ecosystem.
\item Addressing regulators' needs while protecting DeFi values, the Gateway Protocol will benefit all DeFi stakeholders by enabling greater trust and liquidity across the ecosystem.
\end{itemize}